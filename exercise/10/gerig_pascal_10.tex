\documentclass{article}
\usepackage{geometry}
\usepackage{paralist}
\usepackage[T1]{fontenc}
\usepackage{reledmac}
\usepackage{changepage}
\usepackage{amsmath}
\usepackage{scalerel,amssymb}

\usepackage{pgfplots}
\usepackage{tikz}
\usetikzlibrary{positioning}
\usetikzlibrary{shapes.geometric, arrows}
\tikzstyle{arrow} = [thick,->,>=stealth]

\usepackage{fancyhdr}
\fancyhead[L]{
	\begin{tabular}{l}
		\LARGE \textbf{\textsc{Distributed Algorithms}} \\
		\Large Exercise 10
	\end{tabular}
}
\fancyhead[R]{
	\begin{tabular}{r}
		16-104-721 \\
		Pascal \textsc{Gerig}
	\end{tabular}
}
\renewcommand{\headrulewidth}{0.4pt}
\fancyfoot[C]{\thepage}
\renewcommand{\footrulewidth}{0.4pt}

\usepackage{hyperref}

\begin{document}
	\pagestyle{fancy}
	
    \section*{10.1 Byzantine randomized consensus}
    \begin{enumerate}[a)]
        \item \textsc{Optimal resilience for Byzantine consensus:}
        Assume $N = 4$ and $f = 1$, with processes $p_1$ to $p_4$, where $p_1$ is faulty.
        Furthermore assume that $p_2$ proposed $0$, $p_3$ and $p_4$ proposed $1$.
        Lastly assume, that the messages a process sends to itself, are always delayed, such that they are not used for deciding.
        In round 1, $p_3$ might have received  $0$ from $p_1$ and $p_2$ and $1$ from $p_4$, thus $m = 2$, and $v* = 0$, if $coin$ happens to be $0$, then $p_3$ will decide $0$.
        In order to not violate the \textsc{Agreement} Property, $p_2$ and $p_4$ would have to decide $0$ aswell.\\
        $p_4$ might have received $0$ from $p_2$, and $1$ from $p_1$ and $p_3$ in round 1, thus $m = 2$ and $v* = 1$, if $coin$ happens to be $1$, then $p_4$ will decide $1$.
        In this scenario $p_3$ and $p_4$ therefor decide differently, which is clearly violating the \textsc{Agreement} Property mentioned before.
        This proofs, that this algorithm does not achieve a fault tolerance of $N > 3f$.

        \item \textsc{Guru resilience:}
        Assume $N = 5$, $f = 1$, $O = 3$ and $G = 2$ as mentioned in the excercise.
        We have processes $p_1$ to $p_5$, where $p_1$ is the faulty process.
        A process must receive $4$ \textsc{Vote} messages, in order to try to make a decision. 
        Assume, that $p_2$ and $p_3$ have proposed $0$, and $p_4$ and $p_5$ have proposed $1$.
        Further assume, that the messages a process sends to itself, are always delayed, such that they are not used for deciding.
        If $p_2$ has received $0$ from $p_3$ and $1$ from the other three processes, then it is able to decide $1$, if $coin$ equals $1$.
        If $p_5$ has received $0$ from $p_1$ to $p_3$, and $1$ from $p_4$, then it is able to decide $0$, if $coin$ equals $0$.
        These decisions are possible, since $G = 2$ and $m = 3$ in both cases.
        This violates the \textsc{Agreement} Property of randomized byzantine consensus.
        Exactely as in a), this proofs, that the algorithm with given parameters does not achieve a fault tolerance of $N > 4f$.

        \item \textsc{Actual resilience of Gurucoin:}
        In order to be able to violate the \textsc{Agreement} Property, following scenario has to be possible:
        from the correct $N - f$ processes, $\lceil  \frac{N - f}{2} \rceil$ propose $0$, the other $\lfloor \frac{N - f}{2} \rfloor$ propose $1$.
        One correct process receives \textsc{[Vote, r, 0]} from $\lceil  \frac{N - f}{2} \rceil + f$ processes and enough \textsc{[Vote, r, 1]}, to totally receive $N - f$ messages.
        Another correct process receives \textsc{[Vote, r, 1]} from $\lfloor  \frac{N - f}{2} \rfloor + f$ processes and enough \textsc{[Vote, r, 0]}, to totally receive $N - f$ messages.
        The two processes now can only decide, if $m \geq G$, thus if\\
        
        $\lceil  \frac{N - f}{2} \rceil + f < N - 3f$\\
        
        then it's not possible, that two processes decide differently. Knowing that $N = kf + 1$, we get\\

        $\lceil  \frac{(k - 1)f + 1}{2} \rceil + f < (k - 3)f + 1$, $k > 4$ (from b)\\

        The smallest k, to satisfy this inequality for all Natural Numbers f, is $k = 9$: \\
        
        \underline{k = 7}:\\
        $\lceil  \frac{6f + 1}{2} \rceil + f = 4f + 1 < 4f + 1$ is clearly wrong\\
        
        \underline{k = 8}:\\
        $\lceil  \frac{7f + 1}{2} \rceil + f < 4f + 1$ does not hold, since\\

        $\lceil  \frac{7f + 1}{2} \rceil \geq \lceil  \frac{6f + 1}{2} \rceil$ (The right part comes from the step before)

        \underline{k = 9}:\\
        $\lceil  \frac{8f + 1}{2} \rceil + f = 5f + 1 < 6f + 1$\\

        $k < 7$ gives similar errors as $k = 7$ and $k = 8$

        Since $\lceil  \frac{(k - 1)f + 1}{2} \rceil + f$ grows slower than $(k - 3)f + 1$, $k > 4$ the \textsc{Agreement} Property must hold, for all $k > 9$

        The other properties clearly hold for all $k > 9$:\\
        \textsc{Probabilistic termination:} Clearly the algorithm will eventually terminate with a probability of 1

        \textsc{Strong validity:} If all processes propose the same value, then $m \geq G$, thus all correct processes will decide this value.

        \textsc{Integrity:} since the value $decided$ is set to $true$ once a decision is made, a correct process can clearly decide no more than once.
    \end{enumerate}
    
    
\end{document}s