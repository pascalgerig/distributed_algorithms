\documentclass{article}
\usepackage{geometry}
\usepackage{paralist}
\usepackage[T1]{fontenc}
\usepackage{reledmac}
\usepackage{changepage}
\usepackage{amsmath}
\usepackage{scalerel,amssymb}

\usepackage{pgfplots}
\usepackage{tikz}
\usetikzlibrary{positioning}
\usetikzlibrary{shapes.geometric, arrows}
\tikzstyle{arrow} = [thick,->,>=stealth]

\usepackage{fancyhdr}
\fancyhead[L]{
	\begin{tabular}{l}
		\LARGE \textbf{\textsc{Distributed Algorithms}} \\
		\Large Exercise 08
	\end{tabular}
}
\fancyhead[R]{
	\begin{tabular}{r}
		16-104-721 \\
		Pascal \textsc{Gerig}
	\end{tabular}
}
\renewcommand{\headrulewidth}{0.4pt}
\fancyfoot[C]{\thepage}
\renewcommand{\footrulewidth}{0.4pt}

\usepackage{hyperref}

\begin{document}
	\pagestyle{fancy}
	
    \section*{8.1 Total-order broadcast using consensus}
    \begin{enumerate}[a)]
        \item The consensus instance decides on one of the proposed sets, meaning that each process will receive the same set of messages.
        A set is in general unordered, thus we would have the guarantee, that all processes deliver the entire set of messages, but we have no guarantee on the orderer
        thus we would not be able to guarantee that the $total order$ property holds.
        \item One alternative to sorting would be, to only $propose$ one message at the time, since if consensus decides one single message, then ordering is not required.
        Thus one could adapt the algorighm as follows (unchanged partes are not listed again):\\

        \begin{tabular}{ccccc}
			\begin{tabular}{|p{40em}|}
				\hline
				\underline{upon} $unordered \neq \emptyset \wedge wait = false$ \underline{do}\\
				\ \ $\textit{wait} \ = \ true$\\
				\ \ \textit{Initialize a new instance c.round of consensus} \\
				\ \ $\textit{message} \ = \ pickOne(\textit{unordered})$ \\
				\ \ trigger  $\langle \textit{c.round, Propose} \mid \ message$\\
				\hline
			\end{tabular}
        \end{tabular}\\
        
        $pickOne$ selects one element of the given set, in this case it does not matter how the element is chosen.
    \end{enumerate}

    \section*{8.3 Replicated register with local read}
    One could just return the local value whenever a read request is started as follows:\\
    \begin{tabular}{ccccc}
        \begin{tabular}{|p{40em}|}
            \hline
            \\
            \underline{upon} $\langle \textit{nnar, Read}\ \rangle$ \underline{do}\\
            \ \ trigger  $\langle \textit{nnar, ReadReturn} \mid val\rangle$\\
            \\
            \hline
        \end{tabular}
    \end{tabular}\\
    
    Everything that is left out remains as in 8.2

    This change doesn't violate the atomicity property of the register, since the state of the replicated state machine contains only one written value.
    Once this value is overwritten, every local read will from that moment on return the new value.

    \section*{8.2 Atomic register as a replicated state machine}
    \begin{tabular}{ccccc}
        \begin{tabular}{|p{40em}|}
            \hline
            \underline{Implements:} \\
            \ \ (N, N) Atomic Register, instance nnar.\\
            \\
            \underline{upon} $\langle \textit{nnar, Init}\ \rangle$ \underline{do}\\
            \ \ $val \ = \ \emptyset$\\
            \ \ $readCount \ = \ 0$\\
            \ \ $writeCount \ = \ 0$\\
            \\
            \underline{upon} $\langle \textit{nnar, Read}\ \rangle$ \underline{do}\\
            \ \ trigger  $\langle \textit{tob, broadcast} \mid READ\rangle$\\
            \\
            \underline{upon} $\langle \textit{nnar, Write} \mid \ v \ \rangle$ \underline{do}\\
            \ \ trigger  $\langle \textit{tob, broadcast} \mid \ $[\textit{WRITE, v}]$ \rangle$\\
            \\
            \underline{upon} $\langle \textit{tob, Deliver} \mid p, \ READ \rangle$ \underline{do}\\
            \ \ trigger  $\langle \textit{pl, Send} \mid p, \ $[\textit{RACK, val}]$ \rangle$\\
            \\
            \underline{upon} $\langle \textit{tob, Deliver} \mid \ p, \ $[\textit{WRITE, v}]$ \rangle$ \underline{do}\\
            \ \ $val \ = \ v$ \\
            \ \ trigger  $\langle \textit{pl, Send} \mid p, \ \textsc{WACK} \rangle$\\
            \\

            \underline{upon} $\langle \textit{pl, Deliver} \mid \ p, \ \textsc{WACK} \ \rangle$ \underline{do}\\
            \ \ $writeCount \ = \ writeCount \ + \ 1$\\
            \ \ \underline{if} $writeCount = |\Pi|$ \underline{do} \\
            \ \ \ $writeCount \ = \ 0$ \\
            \ \ \ trigger $\langle \textit{nnar, WriteReturn} \rangle$\\
            \\
            \underline{upon} $\langle \textit{pl, Deliver} \mid \ p, \ $ [\textit{RACK, v}] $ \rangle$ \underline{do}\\
            \ \ $readCount \ = \ readCount \ + \ 1$\\
            \ \ \underline{if} $readCount = |\Pi|$ \underline{do} \\
            \ \ \ $readCount \ = \ 0$ \\
            \ \ \ trigger $\langle \textit{nnar, ReadReturn} \mid v \rangle$\\
            \\
            \hline
        \end{tabular}
    \end{tabular}\\

    \begin{itemize}
        \item Termination: Is trivial as clearly both the $ReadReturn$ and the $WriteReturn$ will eventually be triggered
        \item Atomicity: Every possible execution is linearizable, if we use the moment where $val$ is written as linearization points for the $write$ operations.
        once a concurrently beeing written value is saved in $val$ the old value is lost and can not be returned anymore.
        Therefor from that moment on the new value will always be returned.
    \end{itemize}
    
\end{document}