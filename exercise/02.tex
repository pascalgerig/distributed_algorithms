\documentclass{article}
\usepackage{geometry}
\usepackage{paralist}
\usepackage[T1]{fontenc}
\usepackage{reledmac}
\usepackage{changepage}

\usepackage{pgfplots}
\usepackage{tikz}
\usetikzlibrary{positioning}
\usetikzlibrary{shapes.geometric, arrows}
\tikzstyle{arrow} = [thick,->,>=stealth]

\usepackage{fancyhdr}
\fancyhead[L]{
	\begin{tabular}{l}
		\LARGE \textbf{\textsc{Distributed Algorithms}} \\
		\Large Exercise 02
	\end{tabular}
}
\fancyhead[R]{
	\begin{tabular}{r}
		16-104-721 \\
		Pascal \textsc{Gerig}
	\end{tabular}
}
\renewcommand{\headrulewidth}{0.4pt}
\fancyfoot[C]{\thepage}
\renewcommand{\footrulewidth}{0.4pt}

\usepackage{hyperref}

\begin{document}
	\pagestyle{fancy}
	
    \section*{2.1 Pizza or Pasta}
    
        Events:
        \begin{enumerate}
            \item A cook asking Alice to hand him max. `count` jobs: <Cook, getJob | count>
            \item Alice telling a cook which jobs to do: <Alice, deliverJob | jobs, Cook>
            \item A customer ordering a dish: <Customer, orderDish | menu>
            \item A cook handing a dish back to Alice: <Cook, returnDish | menu>
            \item Alice serving the `menu` to the `customer`: <Alice, serveDish | menu, Customer>
            \item Loops the cook to prepare all dishes: <Cook, prepareDish | >
        \end{enumerate}

        Properties:
        \begin{enumerate}
            \item every customer will eventually get his/her order
            \item no dish is delivered if it was not ordered  
            \item every customer gets each order exactly once
        \end{enumerate}


        \begin{adjustwidth}{-3em}{}
          \begin{tabular}{lllll}
            \begin{tabular}{|p{9cm}|}
              \hline
              \textbf{Module \textsc{Alice}:} \\
              \hline
              \underline{Init:} \\
              \ \ $\textit{bufferBob} \ = \ \emptyset$ \\
              \ \ $\textit{bufferCarol} \ = \ \emptyset$ \\
              \ \ $\textit{pizzaBuffer} \ = \ \emptyset$ \\
              \ \ $\textit{bufferOrders} \ = \ \emptyset$ \\
              \\
              \underline{upon} $\langle \textit{Bob, getJob} \mid \textsc{count} \rangle$ \underline{do}\\
              \ \ \underline{if} count < 3
              \ \ \ \underline{while} ($bufferBob == \emptyset$) \{\} \\
              \ \ \ $\textit{jobs} \leftarrow max. first count elements of bufferBob$ \\
              \ \ \ trigger $\langle \textit{Alice, deliverJob} \mid jobs, Bob \rangle$ \\
              \\
              \underline{upon} $\langle \textit{Carol, getJob} \mid \textsc{count} \rangle$ \underline{do}\\
              \ \ \underline{if} count < 7 \\
              \ \ \ \underline{while} ($bufferCarol == \emptyset$) \{\} \\
              \ \ \ $\textit{jobs} \leftarrow max. first count elements of bufferCarol$ \\
              \ \ \ trigger $\langle \textit{Alice, deliverJob} \mid jobs, Carol \rangle$ \\
              \\
              \underline{upon} $\langle \textit{Customer, orderDish} \mid \textit{menu} \rangle$ \underline{do} \\
              \ \ $\textit{bufferOrders} \leftarrow bufferOrders \cup (Customer, menu)$ \\
              \ \ \underline{if} (\textit{menu} == pasta) \\
              \ \ \ \ $\textit{bufferCarol} \leftarrow bufferCarol \cup menu$ \\
              \ \ \underline{else} \\
              \ \ \ \ $\textit{bufferBob} \leftarrow bufferBob \cup menu$ \\
              \\
              \underline{upon} $\langle \textit{Bob, returnDish} \mid \textit{menu} \rangle$ \underline{do} \\
              \ \ $\textit{customer} \leftarrow customer from a tuple of bufferOrders such that tupels menu equals param menu$ \\
              \ \ trigger $\langle \textit{Alice, serveDish} \mid \textit{menu, customer} \rangle$ \\
              \\
              \underline{upon} $\langle \textit{Carole, returnDish} \mid \textit{menu} \rangle$ \underline{do} \\
              \ \ $\textit{customer} \leftarrow customer from a tuple of bufferOrders such that tupels menu equals param menu$ \\
              \ \ trigger $\langle \textit{Alice, serveDish} \mid \textit{menu, customer} \rangle$ \\
              \hline
            \end{tabular}
            &
            &
            \begin{tabular}{l}
              \begin{tabular}{|l|}
                \hline
                \textbf{Module \textsc{Bob}:} \\
                \hline
                \underline{Init:} \\
                \ \ $\textit{buffer} \ = \ \emptyset$ \\
                \ \ $\textit{getJobInProgress} \ = \ false$ \\
                \ \ trigger $\langle \textit{Bob, prepareDish} \mid \textit{} \rangle$ \\
                \\
                \underline{upon} $\langle \textit{Alice, deliverJob} \mid \textit{job, Bob} \rangle$ \underline{do} \\
                \ \ $\textit{getJobInProgress} \ = \ true$ \\
                \ \ $\textit{buffer} \leftarrow buffer \cup job$ \\
                \\
                \underline{upon} $\langle \textit{Bob, prepareDish} \mid \textit{} \rangle$ \underline{do} \\
                \ \ while (true) \\
                \ \ \ \underline{if} $|buffer| < 3 \ \& \ !getJobInProgress$ \\
                \ \ \ \ $\textit{getJobInProgress} \ = \ false$ \\
                \ \ \ \ trigger $\langle \textit{Carol, getJob} \mid \textit{3 - |buffer|} \rangle$ \\
                \ \ \ while ($\textit{buffer} \ == \ \emptyset$) \{\} \\
                \ \ \ $\textit{order} \leftarrow firstElement \ of \ \textit{buffer}$\\
                \ \ \ prepare meal with order\\
                \ \ \ trigger $\langle \textit{Carol, retrunDish} \mid \textit{menu} \rangle$ \\
                \hline
              \end{tabular}
              \\
              \\
              \begin{tabular}{|l|}
                \hline
                \textbf{Module \textsc{Carol}:} \\
                \hline
                \underline{Init:} \\
                \ \ $\textit{buffer} \ = \ \emptyset$ \\
                \ \ $\textit{getJobInProgress} \ = \ false$ \\
                \ \ trigger $\langle \textit{Carol, prepareDish} \mid \textit{} \rangle$ \\
                \\
                \underline{upon} $\langle \textit{Alice, deliverJob} \mid \textit{job, Carol} \rangle$ \underline{do} \\
                \ \ $\textit{getJobInProgress} \ = \ true$ \\
                \ \ $\textit{buffer} \leftarrow buffer \cup job$ \\
                \\
                \underline{upon} $\langle \textit{Carol, prepareDish} \mid \textit{} \rangle$ \underline{do} \\
                \ \ while (true) \\
                \ \ \ \underline{if} $|buffer| < 7 \ \& \ !getJobInProgress$ \\
                \ \ \ \ $\textit{getJobInProgress} \ = \ false$ \\
                \ \ \ \ trigger $\langle \textit{Carol, getJob} \mid \textit{7 - |buffer|} \rangle$ \\
                \ \ \ while ($\textit{buffer} \ == \ \emptyset$) \{\} \\
                \ \ \ $\textit{order} \leftarrow firstElement \ of \ \textit{buffer}$\\
                \ \ \ prepare meal with order\\
                \ \ \ trigger $\langle \textit{Carol, retrunDish} \mid \textit{menu} \rangle$ \\
                \hline
              \end{tabular}
            \end{tabular}
          \end{tabular}			
        \end{adjustwidth}


	\newpage
	\section*{2.2 Safety and liveness}
    \begin{adjustwidth}{2em}{2em}
        Definition Safety Property:\\
        If a property P has been violated in some execution E, then there exists a prefix E' of E such that in every extension
        of E', the property P is violated.
        \\ \\
        Definition Liveness Property:\\
        Property P can be satisfied by some extension E' of a given execution E\\

        
		\begin{enumerate}[(a)]
			\item \textit{If some general attacks at time $t$, then the other general attacks at the same time.} \\
			\textit{\textbf{safety property}} $\rightarrow$ if we asume the property was violated as described in ex. b), then every extension of an execution up to the point where m2 is delivered after t, will result in A attacking after B and not at the same time 
			\item \textit{If $m_2$ arrives after time $t$, then General A attack after General B.} \\
			\textit{\textbf{safety property}} $\rightarrow$ descirbes a violation of the property described in a), see argumentation in a)
			\item \textit{Eventually, General B will attack.} \\
            \textit{\textbf{liveness property}}$\rightarrow$ given any partially completed execution E were both messages were delivered, we can always extend E to E', such that this property will hold.
            \textit{\textbf{safetry property}}$\rightarrow$ fiven an prefix of an execution, where m1 is not delivered, then both Generals (and in paritcular B) will not attack in any extension of the prefix\\
            $\rightarrow$ \textit{\textbf{mixture}}
			\item \textit{If messenegers $m_1$ and $m_2$ are not intercepted, then eventually both generals attack.} \\
			\textit{\textbf{liveness property}}$\rightarrow$ given an Execution E were m1 and m2 were not intercepted and arrived at the receiver, then we can extend E to E' such that both Generals will have launched their Attack.
			\item \textit{If $m_1$ and $m_2$ are not intercepted, then eventually both generals attack at time $t$.} \\
			\textit{\textbf{mixture}}$\rightarrow$ combination of a), because both should attack at same time and d), because both should eventually attack.
		\end{enumerate}			
	\end{adjustwidth}
	
	\section*{2.3 Unreliable clocks}
	\begin{adjustwidth}{2em}{2em}
		\begin{enumerate}[(a)]
			\item \textit{Find two examples, where timing issues lead to safety violations.} \\
            \begin{enumerate}[\textbullet]
                \item General problem, but generals use clocks synchronized with NTP, a firewall blocks connection from NTP server to one generals clock, therefor his time is wrong -> even though they do everything correctly, the won't attack at the same time due to wrong time on one clock.
                \item Assume we have a protocol that calculates some value per time, for instance cost per time of an action. Lets assume this protocol uses a time of day clock.
                Due to a synchronization issue with the NTP server, the time jumps back such that the start time of the calculation equals the end time of the calculation.
                At this point every extension of the described execution will divide by zero causing the protocol to crash.
            \end{enumerate}
			\item \textit{Find two examples, where timing issues lead to liveness volations.} \\
			\begin{enumerate}[\textbullet]
                \item Online shooter, two players are in a battle, player A tries to shoot player B. The server evaluates actions with the given timestamp from each player. if clock of player A is too different from clock of server, then A can hit his oponent on his screen, but the server thinks A missed B, therefor A will not be able to kill player B even though he should be.
                \item Assume a system where the user can change the hardware time and sets it to 1.1.1970 00:00, this will lead to a timestamp of 0. If the system performs operations assuming the timestamp to be a positiv number, it might end up in an infinite loop, not leading to progress at all.
                Fun fact: older iphones had a similar issue
            \end{enumerate}
		\end{enumerate}			
	\end{adjustwidth}
\end{document}