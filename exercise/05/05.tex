\documentclass{article}
\usepackage{geometry}
\usepackage{paralist}
\usepackage[T1]{fontenc}
\usepackage{reledmac}
\usepackage{changepage}
\usepackage{amsmath}
\usepackage{scalerel,amssymb}

\usepackage{pgfplots}
\usepackage{tikz}
\usetikzlibrary{positioning}
\usetikzlibrary{shapes.geometric, arrows}
\tikzstyle{arrow} = [thick,->,>=stealth]

\usepackage{fancyhdr}
\fancyhead[L]{
	\begin{tabular}{l}
		\LARGE \textbf{\textsc{Distributed Algorithms}} \\
		\Large Exercise 05
	\end{tabular}
}
\fancyhead[R]{
	\begin{tabular}{r}
		16-104-721 \\
		Pascal \textsc{Gerig}
	\end{tabular}
}
\renewcommand{\headrulewidth}{0.4pt}
\fancyfoot[C]{\thepage}
\renewcommand{\footrulewidth}{0.4pt}

\usepackage{hyperref}

\begin{document}
	\pagestyle{fancy}
	
    \section*{5.1 Regular register executions}
    \begin{enumerate}[a)]
        \item
        \begin{enumerate}[1.]
            \item With a Read-One Write-All algorithm the execution shown in Figure 1 is not possible.
            each Reader Process reads the value stored in its own Process. 
            As $\textit{read()} \rightarrow y$ happens before $\textit{read()} \rightarrow x$ we know that the stored value in q when starting the second \textit{read()} must be \textit{y}.
            Since we have a (1, N) register all the not mentioned processes are reader processes and can not write the value written in \textit{q}. Therefor the last \textit{read()} will necessarily return y.
            \item We have one write process p, a reader process q and two processes s1 and s2 initial-
            ized with the storage of ?. First p will call the write operation which because it is
            not concurrent will be acknowledged and s1 and s2 store the yalue of x. Therefore
            the read operation which is sequential after the write operation and will therefore
            return x because both s1 and s2 (so it does not matter which of them is read) have
            stored this value. Before the next read call from q is fnished p call the write oper-
            ation to store the value of y. We assume q is trying to read from s1. s1 is storing
            the value of y before the read process is terminating and is therefore returning y.
            The next read operation will read from s2. Because s2 is a little bit slower to write
            the new value it will still have stored the value of x at the time it returns the value
            for the read operation. Therefore the read operation will output x. In the end s2
            will acknowledge that it has stored the value of y.
        \end{enumerate}
        \item We asume that we have three processes: A writer \textit{p} and two readers \textit{q} and \textit{r}.
        \textit{r} does not execute any \textit{read()} operations, but participates in storing the current value.
        Assume further that, after the first \textit{read()} of \textit{q}, all three processes contain the value \textit{x}.
        \textit{p} now starts \textit{write(y)}, at a given time $\textit{t}_1$, \textit{p} has acknowledged \textit{y} while the other two still contain \textit{x}.
        \textit{q} now executes the second \textit{read()}. The majority, that whitnesses the value is $\{ p, q \}$
        Since the \textit{y} stored in \textit{p} has a higher timestamp, than the \textit{x} in \textit{q}, the \textit{read()} will return \textit{y}.
        While the \textit{write()} does not make any progress for a while, \textit{q} at a later time $\textit{t}_2$ starts its last \textit{read()}.
        In this case, the majority whitnessing the value might be $\{q, r\}$.
        None of these two processes has written \textit{y} yet, therefor the function will return \textit{x}
    \end{enumerate}
    \section*{5.2 Read-all write-one regular register}
    \begin{tabular}{ccccc}
        \begin{tabular}{|p{40em}|}
            \hline
            \underline{Implements:} \\
            \ \ (1, N ) -RegularRegister, instance onrr.\\
            \\
            \underline{Uses:} \\
            \ \ BestEffortBroadcast, instance beb; \\
            \ \ PerfectPointToPointLinks, instance pl; \\
            \ \ PerfectFailureDetector, instance P . \\
            \\
            \underline{upon} $\langle \textit{onrr, Init}\rangle$ \underline{do}\\
            \ \ $\textit{val} \ = \ \bot$ \\
            \ \ $\textit{correct} \ = \ \Pi$ \\
            \ \ $\textit{readList} \ = \ []$ \\
            \ \ $\textit{rid} \ = \ 0$ \\
            \ \ $\textit{wts} \ = \ 0$ \\
            \\
            \underline{upon} $\langle \textit{P, Crash} \mid \textsc{p} \rangle$ \underline{do}\\
            \ \ $\textit{correct} \ = \ correct \setminus \{p\}$ \\
            \\
            \underline{upon} $\langle \textit{onrr, Read} \rangle$ \underline{do}\\
            \ \ $\textit{rid} \ = \ rid \ + \ 1$ \\
            \ \ trigger  $\langle \textit{beb, Broadcast} \mid \ [\textsc{READ,} \ rid] \rangle$\\
            \\
            \underline{upon} $\langle \textit{onrr, Write} \mid \ v \ \rangle$ \underline{do}\\
            \ \ $\textit{val} \ = \ v$ \\
            \ \ $\textit{wts} \ = \ wts \ + \ 1$ \\
            \ \ trigger  $\langle \textit{onrr, WriteReturn} \rangle$\\
            \\
            \underline{upon} $\langle \textit{beb, Deliver} \mid \ q, \ [\textsc{READ,} \ rid \ ] \rangle$ \underline{do}\\
            \ \ trigger  $\langle \textit{pl, Send} \mid \textit{q, [VALUE, rid, wts, val]} \rangle$\\
            \\
            \underline{upon} $\langle \textit{pl, Deliver} \mid \ p, \ [ \ \textsc{VALUE,} \ id, \ ts, \ v] \rangle \ s.t. \ id \ = \ rid$ \underline{do}\\
            \ \ $\textit{readList}\leftarrow (\textsc{ts, v})$ \\
            \ \ \underline{if} $\#readlist \ \geq \ \#correct$ \\
            \ \ \ $\textit{value} \ = \ \textit{highestval(readList)}$ \\
            \ \ \ $\textit{readList} \ = \ []$ \\
            \ \ \ trigger $\langle \textit{onrr, ReadReturn} \mid \textit{value} \rangle$\\\\
            \hline
        \end{tabular}
    \end{tabular}
    \\
    \\
    were highestval(x) is defined as in CGR11 Page 148.

    \section*{5.3 (1, 1) Atomic register}
    A reader process can update its $(wts, val)$ pair as the last step before triggering the \textit{ReadReturn}, after the if condition is fulfilled. 
    After the value with the highest timestamp is chosen the process can update its pair with these values so it is up to date again.
\end{document}