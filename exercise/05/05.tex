\documentclass{article}
\usepackage{geometry}
\usepackage{paralist}
\usepackage[T1]{fontenc}
\usepackage{reledmac}
\usepackage{changepage}
\usepackage{amsmath}
\usepackage{scalerel,amssymb}

\usepackage{pgfplots}
\usepackage{tikz}
\usetikzlibrary{positioning}
\usetikzlibrary{shapes.geometric, arrows}
\tikzstyle{arrow} = [thick,->,>=stealth]

\usepackage{fancyhdr}
\fancyhead[L]{
	\begin{tabular}{l}
		\LARGE \textbf{\textsc{Distributed Algorithms}} \\
		\Large Exercise 05
	\end{tabular}
}
\fancyhead[R]{
	\begin{tabular}{r}
		16-104-721 \\
		Pascal \textsc{Gerig}
	\end{tabular}
}
\renewcommand{\headrulewidth}{0.4pt}
\fancyfoot[C]{\thepage}
\renewcommand{\footrulewidth}{0.4pt}

\usepackage{hyperref}

\begin{document}
	\pagestyle{fancy}
	
    \section*{5.1 Regular register executions}
    \begin{enumerate}[a)]
        \item
        \begin{enumerate}[1.]
            \item With a Read-One Write-All algorithm the execution shown in Figure 1 is not possible.
            each Reader Process reads the value stored in its own Process. 
            As $\textit{read()} \rightarrow y$ happens before $\textit{read()} \rightarrow x$ we know that the stored value in q when starting the second \textit{read()} must be \textit{y}.
            Since we have a (1, N) register all the not mentioned processes are reader processes and can not write the value written in \textit{q}. Therefor the last \textit{read()} will necessarily return y.
            \item An execution as shown in Figure 2 however is possible.
	 The key is the "Write one" part. up to the point where the first $\textit{read()} \rightarrow x$ completes, the exetuion is trivial.
	 During the period, in which $\textit{write(y)}$ is executed, we assume that the following things happen in provided order:
	 First of, the \textit{val} of \textit{r} is set to \textit{y}, then the reading of \textit{val} happens in \textit{r} which determines the result of the read operation in \textit{r}.
	 Next, \textit{val} of \textit{q} is read, since the new value \textit{y} has not yet been written, the read operation in \textit{q} must therefor return \textit{x}.
	 Finally \textit{val} of \textit{q} is set to the new value y and then the concurrent write operation will eventually complete.
	 This scenario clearly results in an execution as shown in Figure 2.
        \end{enumerate}
        \item Reproducing the Execution in Figure 1, we asume that we have three processes: A writer \textit{p} and two readers \textit{q} and \textit{r}.
        \textit{r} does not execute any \textit{read()} operations, but participates in storing the current value.
        Assume further that, after the first \textit{read()} of \textit{q}, all three processes contain the value \textit{x}.
        \textit{p} now starts \textit{write(y)}, at a given time $\textit{t}_1$, \textit{p} has acknowledged \textit{y} while the other two still contain \textit{x}.
        \textit{q} now executes the second \textit{read()}. The majority, that whitnesses the value is $\{ p, q \}$
        Since the \textit{y} stored in \textit{p} has a higher timestamp, than the \textit{x} in \textit{q}, the \textit{read()} will return \textit{y}.
        While the \textit{write()} does not make any progress for a while, \textit{q} at a later time $\textit{t}_2$ starts its last \textit{read()}.
        In this case, the majority whitnessing the value might be $\{q, r\}$.
        None of these two processes has written \textit{y} yet, therefor the function will return \textit{x}.\\
        Considering Figure 2, everything would stay the same, except that the $\textit{read()}\rightarrow y$ would be executed on r instead of on q.
    \end{enumerate}

      \section*{5.3 (1, 1) Atomic register}
   If we assume, that we know which process is the reader process, we could modify Algorithm 4.2 such that the reader process stores the latest read value.
   Algorithm 4.2 would be executed as usual, but before triggering the \textit{ReadReturn}, the algorithm would compare the timestamp of the value given by the read value of the regular 4.2 and the latest returned value, and set the latest value and timestamp to whichever of the two has the higher timestamp. Then it could return the value of the two with the higher timestamp which would make the algorithm implement a (1, 1) Atomic Register.

    \section*{5.2 Read-all write-one regular register}
    \begin{tabular}{ccccc}
        \begin{tabular}{|p{40em}|}
            \hline
            \underline{Implements:} \\
            \ \ (1, N ) -RegularRegister, instance onrr.\\
            \\
            \underline{Uses:} \\
            \ \ BestEffortBroadcast, instance beb; \\
            \ \ PerfectPointToPointLinks, instance pl; \\
            \ \ PerfectFailureDetector, instance P . \\
            \\
            \underline{upon} $\langle \textit{onrr, Init}\rangle$ \underline{do}\\
            \ \ $\textit{val} \ = \ \bot$ \\
            \ \ $\textit{correct} \ = \ \Pi$ \\
            \ \ $\textit{readList} \ = \ []$ \\
            \ \ $\textit{rid} \ = \ 0$ \\
            \ \ $\textit{wts} \ = \ 0$ \\
            \\
            \underline{upon} $\langle \textit{P, Crash} \mid \textsc{p} \rangle$ \underline{do}\\
            \ \ $\textit{correct} \ = \ correct \setminus \{p\}$ \\
            \\
            \underline{upon} $\langle \textit{onrr, Read} \rangle$ \underline{do}\\
            \ \ $\textit{rid} \ = \ rid \ + \ 1$ \\
            \ \ trigger  $\langle \textit{beb, Broadcast} \mid \ [\textsc{READ,} \ rid] \rangle$\\
            \\
            \underline{upon} $\langle \textit{onrr, Write} \mid \ v \ \rangle$ \underline{do}\\
            \ \ $\textit{val} \ = \ v$ \\
            \ \ $\textit{wts} \ = \ wts \ + \ 1$ \\
            \ \ trigger  $\langle \textit{onrr, WriteReturn} \rangle$\\
            \\
            \underline{upon} $\langle \textit{beb, Deliver} \mid \ q, \ [\textsc{READ,} \ rid \ ] \rangle$ \underline{do}\\
            \ \ trigger  $\langle \textit{pl, Send} \mid \textit{q, [VALUE, rid, wts, val]} \rangle$\\
            \\
            \underline{upon} $\langle \textit{pl, Deliver} \mid \ p, \ [ \ \textsc{VALUE,} \ id, \ ts, \ v] \rangle \ s.t. \ id \ = \ rid$ \underline{do}\\
            \ \ $\textit{readList}\leftarrow (\textsc{ts, v})$ \\
            \ \ \underline{if} $\#readlist \ \geq \ \#correct$ \\
            \ \ \ $\textit{value} \ = \ \textit{highestval(readList)}$ \\
            \ \ \ $\textit{readList} \ = \ []$ \\
            \ \ \ trigger $\langle \textit{onrr, ReadReturn} \mid \textit{value} \rangle$\\\\
            \hline
        \end{tabular}
    \end{tabular}
    \\
    \\
    were highestval(x) is defined as in CGR11 Page 148.

  
\end{document}