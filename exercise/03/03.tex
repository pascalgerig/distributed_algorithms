\documentclass{article}
\usepackage{geometry}
\usepackage{paralist}
\usepackage[T1]{fontenc}
\usepackage{reledmac}
\usepackage{changepage}
\usepackage{amsmath}
\usepackage{scalerel,amssymb}

\usepackage{pgfplots}
\usepackage{tikz}
\usetikzlibrary{positioning}
\usetikzlibrary{shapes.geometric, arrows}
\tikzstyle{arrow} = [thick,->,>=stealth]

\usepackage{fancyhdr}
\fancyhead[L]{
	\begin{tabular}{l}
		\LARGE \textbf{\textsc{Distributed Algorithms}} \\
		\Large Exercise 03
	\end{tabular}
}
\fancyhead[R]{
	\begin{tabular}{r}
		16-104-721 \\
		Pascal \textsc{Gerig}
	\end{tabular}
}
\renewcommand{\headrulewidth}{0.4pt}
\fancyfoot[C]{\thepage}
\renewcommand{\footrulewidth}{0.4pt}

\usepackage{hyperref}

\begin{document}
	\pagestyle{fancy}
	
    \section*{3.1 Relations among failure detectors}
        Idea: If a process detects a failed process with NPFD, we broadcast this information to all other alive processes.\\
        \\
        \begin{tabular}{ccccc}
        \begin{tabular}{|p{40em}|}
            \hline
            \underline{Init:} \\
            \ \ start timer $\Delta$ \\
            \ \ $\textit{alive} \ = \ \Pi$ \\
            \ \ $\textit{detected} \ = \ \emptyset$ \\
            \\
            \underline{upon} \textit{timeout} \underline{do}\\
            \ \ \underline{for} all $p \ \in \ \textit{alive}$ \\
            \ \ \ \underline{if} \textit{p} failed in last \textit{NPFD} run \\
            \ \ \ \ trigger $\langle \textit{NPFD, detectedFailure} \mid p \rangle$ \\
            \ \ restart timer $\Delta$\\
            \ \ run \textit{NPFD}\\
            \\
            \underline{upon} $\langle \textit{NPFD, detectedFailure} \mid \textsc{failedProcess} \rangle$ \underline{do}\\
            \ \ $\textit{alive} \ \leftarrow \ \textit{alive} \ \setminus \ \textit{failedProcess}$ \\
            \ \ $\textit{detected} \ \leftarrow \ \textit{detected} \ \cup \ \textit{failedProcess}$ \\
            \hline
        \end{tabular}
        \end{tabular}			


    \section*{3.2 Perfect Failure Detector}
    \begin{enumerate}[(a)]
        \item \underline{Implementation:} \\
        \begin{tabular}{lllll}
            \begin{tabular}{p{17cm}}
                %\hline
                \underline{Init:} \\
                \ \ $\textit{alive} \ = \ \Pi$ \\
                \ \ $\textit{detected} \ = \ \emptyset$ \\
                \ \ \underline{for} all $\textit{p} \ \in \ \textit{alive}$ \underline{do} \\
                \ \ \ start timer $\Delta \ + \ \Phi \ + \ 1$ \\
                \ \ start timer $\Delta \ + \ \Phi$ for this process \\
                \\
                \underline{upon} \textit{timeout} for p not this process \underline{do}\\
                \ \ trigger $\langle \textit{P, Crash} \mid p \rangle$ \\
                \ \ $\textit{alive} \ \leftarrow \ \textit{alive} \setminus \ \textit{p}$ \\
                \ \ $\textit{detected} \ \leftarrow \ \textit{detected} \ \cup \ \{ \textit{p} \}$ \\
                \\
                \underline{upon} \textit{timeout} for this process \underline{do}\\
                \ \ reset timer for this process\\
                \ \ trigger $\langle \textit{P, heartbeat} \mid this \rangle$ \\
                \\
                \underline{upon} $\langle \textit{P, heartbeat} \mid \textsc{p} \rangle$ \underline{do}\\
                \ \ wait until timer for \textit{p} = 1 \\
                \ \ reset timer for \textit{p} \\
                %\hline
            \end{tabular}
        \end{tabular}
        \item Every process triggers a heartbeat every $\Delta \ + \ \Phi$, this heartbeat takes max. $\Delta \ + \ \Phi$ to arrive at every other alive process. Therefor if after $\Delta \ + \ \Phi \ + \ 1$, no heartbeat arrives, the not received heartbeat is interpreted as the other process havin crashed. 
        At the same moment every process triggers a new heartbeat and it starts over again.\\
        On the other hand Algorithm 2.5 does not work with heartbeats but with request/response instead.		
    \end{enumerate}
	
	\section*{3.3 Quorum Systems}
	\begin{adjustwidth}{2em}{2em}
		\begin{enumerate}[\footnotesize{\textbullet}]
			\item \textsc{\textbf{Singleton}} \\
			There mustn't be any failed process. Because:
			\[
				\not\exists p \in \Pi : (\not\exists Q \in \mathbb{Q}: p \in Q)
            \]
            Essentially for n = 1 the only existing process is in Q, therefor it is needed and can't fail.
			\item \textsc{\textbf{Majority:}} \\
			\textsc{maximum/minimum:} $\lfloor \frac{n-1}{2} \rfloor$, because:
			\[
				\forall Q \in \mathbb{Q}: \ \mid Q \mid \ = \ \lceil \frac{n+1}{2} \rceil $$ $$
				\Rightarrow \ \mid \Pi \mid - \mid Q \mid \ = \ n - \lceil \frac{n+1}{2} \rceil \ = \ \lfloor \frac{n-1}{2} \rfloor
            \]
			\item \textsc{\textbf{Grid:}}
			\begin{enumerate}[]
				\item \textsc{maximum:} \\
				We take the $Q$ with the fewest elements, which would be equal to the bottom row of the grid (as every Q has one entire row, but as this is the last row, there are no elements underneath). Therefore we must have $k$ correct processes and at most $k^{2} - k$ faulty processes.
				\item \textsc{minimum:} \\
				We take the $Q$ with the most elements which would be equal to the top row with $k-1$ additional processes underneath. Therefore we must have in total $2k-1$ correct processes, so at most $k^{2} - 2k +1$ faulty processes.
			\end{enumerate}
		\end{enumerate}
	\end{adjustwidth}
\end{document}